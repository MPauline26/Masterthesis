\chapter*{Abstract}

This master thesis delves into the domain of credit risk assessment by employing the Random Forest algorithm for estimating the Probability of Default. The study aims to enhance the accuracy of credit risk models by leveraging the ensemble functionality of Random Forest, which combines the strength of multiple decision trees. The research utilizes a comprehensive data set of an American mortgage loan portfolio published by the Federal Home Loan Mortgage Corporation comprising financial indicators, borrower characteristics and loan information to train and validate the models.

The research begins by introducing the importance of credit risk assessment and the regulatory framework set by the Basel Committee on Banking Supervision. Subsequently, the study delves into the theoretical basis of the modeling and validation process, taking focus on the logistic regression model and Random Forest algorithm. Empirical research is conducted to compare the performance of both approaches. First, a default flag in line with the Capital Requirements Regulation Article 178(1) is approximated. Then, the data set is preprocessed to handle missing values and outliers followed by the evaluation of the discriminatory power and correlations of all risk factors. A logistic regression model, widely used in the banking industry for its performance and interpretability, is developed as a baseline model. A Random Forest is trained and fine-tuned through iterative hyperparameter tuning for optimal performance.

Results obtained from the predictive models are thoroughly evaluated on out-of-sample data sets using metrics such as the area under the Receiver Operating Characteristic curve, Gini coefficient, F1-score and confusion matrix. The comparison of both approaches reveals that while the Random Forest outperforms the logistic regression model, the improvement in performance is only marginal. 

Finally, the study explores feature importance and other interpretability methods within the Random Forest framework to identify key variables influencing credit risk, increase transparency and get a better understanding of the “Black Box”. The findings of this research contribute to the ongoing discourse on enhancing credit risk modeling techniques and offer valuable insights for participants in the financial industry.