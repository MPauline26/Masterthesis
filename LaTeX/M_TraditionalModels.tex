\chapter{Traditional models}

\section{Overview}
To estimate a PD model, different types of models varying in complexity are available:

\begin{enumerate}
  \item Statistical Models: This type utilizes historical data for the estimation process. Techniques such as logistic regression, survival analysis, and machine learning algorithms are used to predict default events and analyse contributing risk factors.
  \item External Rating Models: Rating agencies develop models that assign credit ratings to borrowers. These models consider various factors, e.g., financial statements and macroeconomic conditions, to evaluate creditworthiness. These type of PD models are only available for a limited portion of borrowers. 
  \item Expert Judgement: In cases where historical data is limited or only a low number of default events is available, expert judgement will become most relevant. Experienced credit analysts rely on their expertise and industry knowledge to estimate the PD based on qualitative factors, market conditions and information of the client.
\end{enumerate}

In practice, a substantial portion of the banking sector employ a combination of multiple types of models in their credit risk assessment.

\section{Logit and Probit regression}
Logistic regression is one of the most commonly used statistical models in the banking industry. It is particularly useful when the dependent variable is binary. The model estimates the probability of default by fitting a link function to the explanatory variables. Therefore it transforms the resulting score, which can take any negative or positive value, to the corresponding PD value ranging between 0 and 1. A high model score means a lower probability to default and vice versa. For the link function the logistic function or standard normal cumulative distribution function can be used, resulting in the logit or probit model respectively (Fig. \ref{fig:tm_logfunc}, \ref{fig:tm_sncdfunc}). An advantage of the logit model is the heavier tails in the logistic distribution, which would therefore put higher weights to extreme events, visible in Fig. \ref{fig:tm_distr}. 

\section{Other models}

\subsection{Linear regression}
During the linear regression, the algorithm estimates a linear relationship between the default variable, which assumes either the value 0 (non default) or 1 (default), and explanatory variables, which can both be continuous and categorical independent variables, for example income, employment duration and profession (Fig. \ref{fig:tm_linreg}, Eq. \ref{eq:tm_linreg}). Unfortunately, due to the binary dependent variable, the residuals are heteroscedastic and therefore the estimation of the coefficients is inefficient. Additionally, the model may output non logical result, like negative values or a PD over 100\%.

\subsection{External Ratings}
Scorings of corporate clients are usually performed mainly by a credit analyst and only partly automated, due to the low number of default events, but also due to the type of information available. If the financial institution does not have enough resources to develop and maintain internal models, external ratings may be used. Most known rating agencies are Standard \& Poor, Moody's and Fitch. They provide ratings for a wide range of corporations, since most companies request a rating before a sale or registration of a debt issue. An analyst will use their financial statements of the last years and additional information to derive a rating, which is then discussed in a rating committee. Afterwards, the corporation is informed about their rating, the corresponding factors, and given the opportunity to respond and finally the ratings will be published. A disadvantage of external ratings observed in the past is the conflict of interest, since the ratings are mainly paid by the company. It is suspected, that good ratings were related to high fees, visible during the financial crisis where many structured bonds with high scorings deteriorated unexpectedly. 

\subsection{Shadow Rating}
The goal in the Shadow Rating approach is to estimate a model, which produces similar PDs as ratings, determined by external rating agencies. For this process, variables, which are possible input factors, need to be defined, for example macroeconomic factors and financial statements. The model's output serves as a valuable tool for credit analysts in making the final decisions. 